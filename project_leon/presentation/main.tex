% \documentclass[10pt]{beamer}
\documentclass[handout, 10pt]{beamer}

\usepackage{homework}

\usetheme{Madrid}
\setbeamersize{text margin left=30pt,text margin right=20pt}
\graphicspath{{../figures/}}

\usepackage[backend=biber,style=authoryear-comp]{biblatex}
\addbibresource{literature.bib}
% Beamer needs this to show footnotes properly
\setbeamertemplate{footnote}{%
  \parindent 0em\noindent
  \raggedright
  \insertfootnotemark\insertfootnotetext\par
}
% Change item symbols for bibliography
\setbeamertemplate{bibliography item}[triangle]
\setbeamertemplate{itemize item}[triangle]

% Title Page
\title[Nonlinear Noise Reduction]{Nonlinear Noise Reduction}
\subtitle{Computational Physics Seminar on Analyzing Biomedical Signals}
\author{Leon Galbas}
\date{12.01.2026}
\institute[Universität Bonn]{Rheinische Friedrich-Wilhelms-Universität Bonn}


\begin{document}
\frame{\titlepage}

% Table of Contents
\begin{frame}{Outline}
    \tableofcontents
\end{frame}
% Current section
\AtBeginSection[ ]
{
\begin{frame}{Outline}
    \tableofcontents[currentsection]
\end{frame}
}
\AtBeginSubsection[]
{
  \begin{frame}{Outline}
    \tableofcontents[currentsection,currentsubsection]
  \end{frame}
}

% Start of actual presentation

\section{Motivation}
\begin{frame}
    \frametitle{Definitions and scope}
    \textbf{So far:}
    \begin{itemize}
        \item Systems evolve deterministically
            \[\vb{y}_{n+1} = f(\vb{y}_n).\]
        \pause
        \item Timeseries are projections of the true state
            \[y_n = s(\vb{y}_n).\]
    \end{itemize}
    
\end{frame}
\begin{frame}
    \frametitle{Definitions and scope}
    \begin{itemize}
        \item Measurement noise:
            \[x_n = s(\vb{y}_n) + \eta_n\]
        \pause
        \item Dynamic noise:
            \[\vb{y}_{n+1} = f(\vb{y}_n + \vb{\eta}_n)\]
    \end{itemize}
    \textbf{In this talk, we only focus on measurement noise.}
\end{frame}

\begin{frame}
    \frametitle{Why does classical filtering not work?}
    \begin{figure}
        \centering
        \includegraphics[width=0.85\textwidth]{plots/01_motivation_henon-noise-ts.pdf}
        \caption*{Timeseries of the $x$-coordinate of the Hénon map for 200 steps (above) and of white noise with the same mean and standard derivation (below).}
    \end{figure}
\end{frame}

\begin{frame}
    \frametitle{Why does classical filtering not work?}
    \begin{figure}
        \centering
        \includegraphics[width=0.85\textwidth]{plots/01_motivation_henon-noisy-ts.pdf}
        \caption*{Timeseries of the Hénon map without noise (above) and with $5\%$ relative noise (below).}
    \end{figure}
\end{frame}

\begin{frame}
    \frametitle{Why does classical filtering not work?}
    \begin{figure}
        \centering
        \includegraphics[width=0.85\textwidth]{plots/01_motivation_henon-noise-power.pdf}
        \caption*{Power spectrum of the Hénon map (above) and of white noise with the same mean and standard derivation (below) for 2000 steps.}
    \end{figure}
\end{frame}

\begin{frame}
    \frametitle{The problem}
    \textbf{Problem:}
    \begin{itemize}
        \item Spectral analysis and filtering are of limited use for reducing noise in chaotic systems. ($\to$ broadband)
    \end{itemize}
    \pause
    \textbf{Idea:}
    \begin{itemize}
        \item Make use of higher dimensional structures in the state space of dynamical systems.
    \end{itemize}
\end{frame}


\section{Recap: Phase Space Reconstruction}
\begin{frame}
    \frametitle{Takens' theorem}
    \begin{itemize}
        \item \textbf{Goal:} Reconstruct the phase space vectors from a scalar timeseries.
        \pause
        \item \textbf{Takens' theorem:} Use delay vectors
            \[ x_n \to \vb{x}_n := (x_n, x_{n+\tau},\ldots,x_{n+(m-1)\tau})\]
    \end{itemize}
\end{frame}

\begin{frame}
    \frametitle{Manifold hypothesis}
    \begin{itemize}
        \item \textbf{Assumption:} Noise-free dynamics $\vb{y}_n$ lie on submanifold $\sA$ with
            \[\dim{\sA} = D < m.\]
        \item Noise scatters points $\vb{x}_n$ away from $\sA$.
    \end{itemize}
\end{frame}

\begin{frame}
    \frametitle{Recap: Hénon map}
    Defined by 
    \begin{align*}
        x_{n+1} &= 1 - ax_n^2 + y_n \\
        y_{n+1} &= bx_n,
    \end{align*}
    with parameters
    \[a=1.4 \qq{and} b=0.3,\]
    and starting value
    \[(x_0,y_0) = (0,0).\]
\end{frame}

\begin{frame}
    \frametitle{Hénon attractor with and without noise}
    \begin{figure}
        \centering
        \includegraphics[width=0.85\textwidth]{plots/02_phasespace_henon-noise-attr.pdf}
        \caption*{Noise-free reconstructed Hénon attractor for 10000 steps (left) and reconstructed phase space from data with $5\%$ relative noise (right).}
    \end{figure}
\end{frame}

\begin{frame}
    \frametitle{General strategy}
    \textbf{Geometric idea:}
    \begin{itemize}
        \item Identify the manifold $\sA$ of the attractor.
        \item Move noisy points $\vb{x}_n$ towards $\sA$.
    \end{itemize}
\end{frame}


\section{Nonlinear Noise Reduction}

\subsection{A simple method}
\begin{frame}
\frametitle{A simple method: local averaging}
Proposed by \cite{schreiber1993extremely}:
\begin{enumerate}
    \item Find neighbourhood $\sU_{n,\eps} = \{k \mid \norm{\vb{x}_k-\vb{x}_n} < \eps\}$ around $\vb{x}_n$.
    \item Compute corrected vectors
        \[ x_{n+m/2}^\f{corr} = \frac{1}{\abs{\sU_{n,\eps}}}\sum_{k\in\sU_{n,\eps}} x_{k+m/2}.\]
    \item Replace all $x_n$.
    \item Decrease $\eps \to \f{RMS}(x_n^\f{corr}-x_n)$.
    \item Iterate.
\end{enumerate}
\end{frame}

\begin{frame}
    \frametitle{Simple noise reduction on the Hénon attractor}
    \begin{figure}
        \centering
        \includegraphics[width=0.85\textwidth]{plots/03a_simple_noisy-reduced-attr.pdf}
        \caption*{Noisy reconstructed Hénon attractor for 10000 steps (left) and attractor after applying the simple noise reduction algorithm for five iterations with $\eps_0=15\%$.}
    \end{figure}
\end{frame}

\begin{frame}
    \frametitle{Limitations}
    \begin{figure}
        \centering
        \includegraphics[height=0.7\textheight]{plots/03a_simple_error.pdf}
        \caption*{\textbf{Curvature bias:} In curved regions, the average pulls points inside the curve, shrinking the attractor.}
    \end{figure}
\end{frame}


\subsection{The projective method}
\begin{frame}
    \frametitle{The projective method}
    \textbf{Assumption:}
    \begin{itemize}
        \item The true $D$-dim attractor $\sA$ is smooth enough to be locally approximated linearly. 
    \end{itemize}
    \pause
    \textbf{Idea:} (\cite{grassberger1993noise})
    \begin{enumerate}
        \item Find (hyper-)plane that locally approximates the attractor in an $\eps$-neighbourhood of $\vb{x}_n$.
        \item Project $\vb{x}_n$ onto that subspace.
        \item Reduce $\eps$.
        \item Iterate.
    \end{enumerate}
\end{frame}

\begin{frame}
    \frametitle{The projective method}
    \begin{figure}
        \centering
        \includegraphics[height=0.7\textheight]{plots/03b_projective_diagram-simple-prod.pdf}
        \caption*{Schematic visualization of how both methods apply noise correction to a point outside the attractor.}
    \end{figure}
\end{frame}

\begin{frame}
    \frametitle{Constraints}
    \textbf{Ideally:}
    \[F(\vb{y}_n) = 0\]
    \textbf{Approximation}:
    \[ \vb{a}^{(n)}\cdot(\vb{x}-\overline{\vb{x}}^{(n)}) \approx 0.\]
\end{frame}

\begin{frame}
    \frametitle{Finding the subspace}
    For a fixed $\vb{x}_n$ with neighbourhood $\sU_n$:
    \begin{enumerate}
        \item Covariance matrix:
            \[C_{ij} = \frac{1}{\abs{\sU_n}}\sum_{k\in\sU_n}x_{k+i}x_{k+j} - \overline{x_{k+i}}\cdot\overline{x_{k+j}}\]
        \vspace{-0.2cm}
        \pause
        \item Pricipal Component Analysis (PCA):
            \[\sigma_1^2 \geq \ldots \geq \sigma_m^2 \text{ eigenvalues with eigenvectors } \vb{v}_1,\ldots,\vb{v}_m\]
        \vspace{-0.2cm}
        \pause
        \item Identify subspace:\begin{itemize}
            \item Large eigenvalues $\rightarrow$ tangent
            \item Small eigenvalues $\rightarrow$ orthogonal
        \end{itemize}
    \end{enumerate}
\end{frame}

\begin{frame}
    \frametitle{Projecting onto the subspace}
    \begin{itemize}
        \item Noise directions (small eigenvalues):
            \[\vb{v}_{D+1},\ldots,\vb{v}_m \leftrightarrow \text{ orthogonal to the attractor}\]
        \pause
        \item Correction:
            \[\delta\vb{x}_n = -\sum_{k=D+1}^m (\vb{x}_n\cdot\vb{v}_k)v_k\]
    \end{itemize}
\end{frame}

\begin{frame}
    \frametitle{The consistency problem}
    \begin{itemize}
        \item The scalar $x_n$ appears in $m$ delay vectors
            \begin{align*}
                \vb{x}_n &= (\textcolor{red}{x_n}, \ldots, x_{n+m-1}) \\
                &\vdots \\
                \vb{x}_{n-m+1} &= (x_{n-m+1},\ldots, \textcolor{red}{x_n}).
            \end{align*}
        \item \textbf{Problem:} Projecting $\vb{x}_n$ independently gives $m$ different corrections for $x_n$.
        \pause
        \item \textbf{Solution:} Average
            \[ x_n^\f{corr} = x_n + \frac{1}{m}\sum_{j=0}^{m-1}\delta x_{n-j}^{(j)}\]
    \end{itemize}
\end{frame}

\begin{frame}
    \frametitle{Projective noise reduction on the Hénon attractor}
    \begin{figure}
        \centering
        \includegraphics[width=0.85\textwidth]{plots/03b_projective_noisy-reduced-attr.pdf}
        \caption*{Noisy reconstructed Hénon attractor for 10000 steps (left) and attractor after applying the first order noise reduction algorithm for three iterations with $m=5$ and $D=2$.}
    \end{figure}
\end{frame}



\section{Results}
\begin{frame}
    \frametitle{Performance of the zero-order algorithm}
    \begin{figure}
        \centering
        \includegraphics[width=0.85\textwidth]{plots/04_results_noisy-reduced-attr_zero.pdf}
        \caption*{Noisy reconstructed Hénon attractor for 10000 steps (left) and attractor after applying the zero order noise reduction algorithm for two iterations with $m=5$ and $D=2$.}
    \end{figure}
\end{frame}

\begin{frame}
    \frametitle{Performance of the zero-order algorithm}
    \begin{figure}
        \centering
        \includegraphics[width=0.85\textwidth]{plots/04_results_noisy-reduced-attr_zoomed_zero.pdf}
        \caption*{Zoomed view of the noise-free reconstructed Hénon attractor for 10000 steps (left) and attractor after applying the zero order noise reduction algorithm for two iterations with $m=2$ and $\eps_0=10\%$.}
    \end{figure}
\end{frame}

\begin{frame}
    \frametitle{Change per iteration of the zero-order algorithm}
    \begin{figure}
        \centering
        \includegraphics[width=0.85\textwidth]{plots/04_results_rms-diff-true_zero.pdf}
        \caption*{RMS difference of the true noise free timeseries $y_n$ and the corrected timeseries $x_n^\f{corr}$ for 10 iterations.}
    \end{figure}
\end{frame}

\begin{frame}
    \frametitle{Performance of the first order algorithm}
    \begin{figure}
        \centering
        \includegraphics[width=0.85\textwidth]{plots/04_results_noisy-reduced-attr.pdf}
        \caption*{Noisy reconstructed Hénon attractor for 10000 steps (left) and attractor after applying the first order noise reduction algorithm for five iterations with $m=5$ and $D=2$.}
    \end{figure}
\end{frame}

\begin{frame}
    \frametitle{Performance of the first order algorithm}
    \begin{figure}
        \centering
        \includegraphics[width=0.85\textwidth]{plots/04_results_noisy-reduced-attr_zoomed.pdf}
        \caption*{Zoomed view of the noise-free reconstructed Hénon attractor for 10000 steps (left) and attractor after applying the first order noise reduction algorithm for five iterations with $m=5$ and $D=2$.}
    \end{figure}
\end{frame}

\begin{frame}
    \frametitle{Change per iteration of the first order algorithm}
    \begin{figure}
        \centering
        \includegraphics[width=0.85\textwidth]{plots/04_results_rms-diff-true.pdf}
        \caption*{RMS difference of the true noise free timeseries $y_n$ and the corrected timeseries $x_n^\f{corr}$ for 10 iterations.}
    \end{figure}
\end{frame}


% Sources
\section*{Sources}
\begin{frame}{Sources: Software}
    The software used to generate the datasets as well as apply the algorithms is the TISEAN package first published by \cite{hegger1999practical}. The plots shown in this presentation were produced using the Python packages Matplotlib and Numpy. The Code can be found in the sources below:
    \begin{itemize}
        \item \textbf{TISEAN:} \url{https://www.mpipks-dresden.mpg.de/tisean/Tisean_3.0.1/index.html}
        \item \textbf{Presentation:} \url{https://github.com/lebega/ws2025-physics650-seminar-analyzing-biomedical-signals}
    \end{itemize}
\end{frame}
\begin{frame}{Sources: Literature}
    \begin{tiny}
        \printbibliography
    \end{tiny}
\end{frame}

\end{document}
