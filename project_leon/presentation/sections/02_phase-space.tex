\begin{frame}
    \frametitle{Takens' theorem}
    \begin{itemize}
        \item \textbf{Goal:} Reconstruct the phase space vectors from a scalar timeseries.
        \pause
        \item \textbf{Takens' theorem:} Use delay vectors
            \[ x_n \to \vb{x}_n := (x_n, x_{n+\tau},\ldots,x_{n+(m-1)\tau})\]
    \end{itemize}
\end{frame}

\begin{frame}
    \frametitle{Manifold hypothesis}
    \begin{itemize}
        \item \textbf{Assumption:} Noise-free dynamics $\vb{y}_n$ lie on submanifold $\sA$ with
            \[\dim{\sA} = D < m.\]
        \item Noise scatters points $\vb{x}_n$ away from $\sA$.
    \end{itemize}
\end{frame}

\begin{frame}
    \frametitle{Recap: Hénon map}
    Defined by 
    \begin{align*}
        x_{n+1} &= 1 - ax_n^2 + y_n \\
        y_{n+1} &= bx_n,
    \end{align*}
    with parameters
    \[a=1.4 \qq{and} b=0.3,\]
    and starting value
    \[(x_0,y_0) = (0,0).\]
\end{frame}

\begin{frame}
    \frametitle{Hénon attractor with and without noise}
    \begin{figure}
        \centering
        \includegraphics[width=0.85\textwidth]{plots/02_phasespace_henon-noise-attr.pdf}
        \caption*{Noise-free reconstructed Hénon attractor for 10000 steps (left) and reconstructed phase space from data with $5\%$ relative noise (right).}
    \end{figure}
\end{frame}

\begin{frame}
    \frametitle{General strategy}
    \textbf{Geometric idea:}
    \begin{itemize}
        \item Identify the manifold $\sA$ of the attractor.
        \item Move noisy points $\vb{x}_n$ towards $\sA$.
    \end{itemize}
\end{frame}